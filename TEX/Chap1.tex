\chapter{Introduction}
\section{Présentation de l'image étudiée et analyse des bandes spectrales}

Dans cette section, nous présentons l'image multispectrale utilisée pour nos expérimentations, ainsi que l'analyse de ses 9 bandes spectrales. L'objectif est de comprendre la répartition des intensités dans les bandes et d'observer leurs variations. \cite{islam2012artificial}

\subsection{Image originale et bandes spectrales}

L'image multispectrale originale contient 9 bandes spectrales distinctes. Chaque bande correspond à une région spécifique du spectre lumineux. La figure ci-dessous illustre les variations d'intensité pour ces bandes.

\begin{figure}[H]
    \centering
    \includegraphics[width=0.8\textwidth]{Imgs/Université_de_Poitiers_Logo.png}
    \caption{Courbes de sensibilité spectrale relative pour les 9 bandes de l'image multispectrale.}
    \label{fig:multispectral_bands}
\end{figure}

\subsection{Analyse des résultats}

Les variations d'intensité spectrale pour chaque bande montrent des différences significatives dans le contenu de l'image. Voici les principales observations :
\begin{itemize}
    \item \textbf{Bande 1 :} La bande 1 présente une intensité élevée au début de l'image et décroît progressivement. Cela peut être attribué à des éléments spécifiques sensibles à cette longueur d'onde.
    \item \textbf{Bande 2 et Bande 3 :} Ces bandes montrent des variations modérées, avec des pics d'intensité associés à certaines zones structurées dans l'image.
    \item \textbf{Bande 4 à Bande 6 :} Ces bandes présentent une intensité plus uniforme, indiquant une bonne couverture spectrale des zones dominantes de l'image.
    \item \textbf{Bande 7 à Bande 9 :} Les bandes finales ont une intensité relativement plus faible, ce qui indique une absorption ou une réflexion moindre dans ces régions spectrales.
\end{itemize}

Ces observations mettent en évidence la richesse de l'information contenue dans l'image multispectrale et justifient l'importance d'analyser les différentes bandes pour une meilleure compréhension des données.
